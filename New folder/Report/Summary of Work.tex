\documentclass[11pt, a4paper]{article}
\usepackage{amsmath}
\usepackage{amssymb}
\usepackage{amsthm}
\usepackage{epsfig}
\usepackage[margin=0.7in]{geometry}
\usepackage{enumitem}
% \usepackage{enumerate}
\usepackage{graphicx}

%%%%%%%%%% Package for Highlight         %%%%%%%
\usepackage[usenames, dvipsnames]{xcolor}
%%%%%%%%%%%%%%%%%%%%%%%%%%%%%%%%%%%%%%%%%%%%%%%%%%


\usepackage{tikz}
\usetikzlibrary{automata, positioning}
\usepackage{caption}
\usepackage{mathtools}
\usepackage{algorithm}
\usepackage{algorithmic}
\usepackage{todonotes}
\usepackage{times}
\usepackage{pdfpages}
\usepackage{mathtools}
%\usepackage[]{algorithm2e}

\usepackage{lipsum} % paragraph spacing
\usepackage{titlesec} % paragraph spacing
\usepackage{comment}

%%%%%%%%%%% CHINESE TYPING  %%%%%%%%%%%%
\usepackage{CJK} 
% How to use: 
% \begin{CJK}{UTF8}{} 
% \end{CJK}
%%%%%%%%%%%%%%%%%%%%%%%%%%%%%%%%%%%%%%%%%%%

%%%%%%%%%%%%%%%%%%%%%%%%%%
% 3 Columns & landscape	 %
%%%%%%%%%%%%%%%%%%%%%%%%%%
%\usepackage{multicol}
%\usepackage{pdflscape}


% For writing code:
\usepackage{fancyvrb}
\DefineVerbatimEnvironment{code}{Verbatim}{fontsize=\small}
\def\showvrb#1{%
	\texttt{\detokenize{#1}}%
}

\linespread{1.0}
\setlength{\parindent}{0pt} % Sets the indent length to 0
\setlength{\parskip}{0pt plus 1pt minus 1pt} % paragraph vertical distance

\everymath{\displaystyle} % displays inline math as displaymath

\hyphenpenalty=10000 % force no hyphenation



\begin{comment}
%%%%%%%%%%%%%%%%%%%%%%%%%%%%%%%%%
% ONLY RECOMMENDED FOR 3 COLUMN %
%%%%%%%%%%%%%%%%%%%%%%%%%%%%%%%%%%%%%%%%%%%%%%%%%%%%%%%%%%%%%%%%%%%%%

\setlist[itemize]{noitemsep, topsep=0pt, leftmargin=0.2in} % compacts lists with no item separation,indent from the left side = 0.2
\titlespacing\section{1pt}{5pt plus 4pt minus 2pt}{2pt plus 2pt minus 2pt}%paragraph spacing
\titlespacing\subsection{0pt}{5pt plus 4pt minus 2pt}{2pt plus 2pt minus 2pt}%paragraph spacing
\titlespacing\subsubsection{0pt}{5pt plus 4pt minus 2pt}{2pt plus 2pt minus 2pt}%paragraph spacing
\end{comment}


%%%%%%%%%%%%%%%% Normally use this! %%%%%%%%%%%%%%%%%%%%%%%%%%%%%%%%%%%%%%%%%%%%%%%%%%%%
%
\setlist[itemize]{noitemsep, topsep=0pt} % compacts lists with no item separation    %%%
%
%
%%%%%%%%%%%%%%%%%%%%%%%%%%%%%%%%%%%%%%%%%%%%%%%%%%%%%%%%%%%%%%%%%%%%%%%%%%%%%%%%%%%%%%%%


\title{Enter title HERE}
\author{Ying He}

%%%%%%%%%%%%%%%%%%%%%%%%%%%%%%%
%   Commonly used theorems    %
%%%%%%%%%%%%%%%%%%%%%%%%%%%%%%%

% Use roman for text - numbering follows onwards
\theoremstyle{definition}
\newtheorem{defn}{Definition}[section]
\newtheorem{them}{Theorem}[section]
\newtheorem{lem}{Lemma}[them]
\newtheorem{prop}[them]{Proposition}
\newtheorem{corr}[them]{Corollary}
\newtheorem{corrr}{Corollary}[them] %for consistency issues
% examples follow different numbering
\newtheorem{eg}{Example}[section]
\newtheorem{egg}[them]{Example}
\newtheorem*{themf}{Theorem 5.0}
\newtheorem*{themm}{Theorem}
\newtheorem{question}{Question}

%%%%%%%%%%%%%
% Shortcuts %
%%%%%%%%%%%%%

\newcommand{\Q}{\mathbb{Q}}
\newcommand{\C}{\mathbb{C}}
\newcommand{\F}{\mathbb{F}}
\newcommand{\R}{\mathbb{R}}
\newcommand{\Z}{\mathbb{Z}}
\newcommand{\N}{\mathbb{N}}
\newcommand{\D}{\mathbb{D}}

\newcommand{\mcC}{\mathcal{C}}
\newcommand{\mcN}{\mathcal{N}}
\newcommand{\mcE}{\mathcal{E}}
\newcommand{\mcP}{\mathcal{P}}
\newcommand{\mcF}{\mathcal{F}}
\newcommand{\mcS}{\mathcal{S}}
\newcommand{\mcA}{\mathcal{A}}
\newcommand{\mcB}{\mathcal{B}}
\newcommand{\mcX}{\mathcal{X}}
\newcommand{\mcL}{\mathcal{L}}
\newcommand{\mcH}{\mathcal{H}}
\newcommand{\mcl}{\mathcal{l}}

\newcommand{\bfe}{\mathbf{e}}
\newcommand{\mbf}{\mathbf}

\newcommand{\ran}{\mbox{Ran}}

\newcommand{\al}{\alpha}
\newcommand{\s}{\sigma}
\newcommand{\ep}{\epsilon}
\newcommand{\dl}{\delta}
\newcommand{\sg}{\sigma}

\newcommand{\bs}{\backslash}
\newcommand{\n}{\\}
\newcommand{\ol}{\overline}
%\newcommand{\bb}{\textbf} %Bold front
%\newcommand{\ul}{\underline} %underline

\newcommand{\ltlU}{\textsf{ \bfseries UNTIL }}
\newcommand{\ltlF}{\textsf{\bfseries F }}
\newcommand{\ltlG}{\textsf{\bfseries G }}
\newcommand{\ltlX}{\textsf{\bfseries X }}
\newcommand{\ltlA}{\textsf{\bfseries A }}
\newcommand{\ltlE}{\textsf{\bfseries E }}

%%%%%%%%%%%%%%%%%%%%%%%%%
% Standard big brackets %
%%%%%%%%%%%%%%%%%%%%%%%%%

\newcommand{\floor}[1]{\left\lfloor{#1}\right\rfloor}
\newcommand{\ceil}[1]{\left\lceil{#1}\right\rceil}
\newcommand{\bbA}[1]{\left({#1}\right)}
\newcommand{\bbB}[1]{\left[{#1}\right]}
\newcommand{\bbC}[1]{\left\{{#1}\right\}}



%%%%%%%%%%%%%%%%%
%  MATRIX       %
%%%%%%%%%%%%%%%%%
\newcommand{\matxx}[9]{\begin{pmatrix} #1 & #2 & #3 \\ #4 & #5 & #6 \\ #7 & #8 & #9 \end{pmatrix}}
\newcommand{\matx}[4]{\begin{pmatrix} #1 & #2 \\ #3 & #4 \end{pmatrix}}
%example
%    $\matx{1}{2}{3}{4}{5}{6}{7}{8}{9}$

%%%%%%%%%%%%%%%%

%%%%%%%%%%%%%%%%%%%%%%%%%%%%%
% Highlight color
%%%%%%%%%%%%%%%%%%%%%%%%%%%

\newcommand{\hgy}{\colorbox{GreenYellow}} % Light Green
\newcommand{\hyg}{\colorbox{YellowGreen}} % another Light Green
\newcommand{\hlb}{\colorbox{SkyBlue}} % light blue
\newcommand{\hpink}{\colorbox{pink}} % Light pink
\newcommand{\hor}{\colorbox{orange}} % Orange
\newcommand{\hlor}{\colorbox{Apricot}} % Light Orange




%%%%%%%%%%%%%%%%%%
% Start Document %
%%%%%%%%%%%%%%%%%%

\begin{document}


\begin{comment}
	\begin{landscape}
	\begin{multicols*}{3}
\end{comment}
\begin{comment}
	\textbf{abcdefg}
	%\tableofcontents
	\hlor{abcd}
	\[\matx{1}{2}{3}{4}\] al;dfjajfa
	woyaozaizhedazi $\matx{1}{2}{3}{4}$
	\[\alpha \cdot \beta \]
	123456789/*
\end{comment}

\title{Summary of Work}
\maketitle

\section{Summary of Work}
	The whole thesis was divided into three sections, which are, namely,
	\begin{itemize}
		\item DesignBuilder Building Modeling
		\item Initial EnergyPlus Building Simulation
		\item SIA Building Energy Calculation
		\item Building Envelope and Weather Condition Calibration
		\item Building Parameters Variation
		\item Heat Island Effect and Global Warming Effect
	\end{itemize}
	
\section{Building Modeling}
	This project focuses on 2 existing uninsulated buildings. One is office building (Sumatrastrasse 10) and the other is residential building (Honggerstrasse 23).\\
	\begin{itemize}
		\item Residential Building and Office Building
			\begin{itemize}
				\item Re-generate building plan from pdf reports
				\item Check SIA382 model for detailed building material and building part sizes
				\item Generate the building model using DesignBuilder
			\end{itemize}
		\item Weather Data
			\begin{itemize}
				\item Obtain standard weather data in Zurich
				\item Obtain weather data (hourly) in year 2015 in Zurich from \textit{IDAWEB-MeteoSwiss}
					\begin{itemize}
						\item Temperature (Dry/Wet bulb)
						\item Relative humidity
						\item Wind direction
						\item Wind speed
						\item (Didn't change radiation)
					\end{itemize}
				\item Generate Zurich year 2015 \textit{.epw} weather file 
			\end{itemize}	
	\end{itemize}

\section{Initial EnergyPlus Building Simulation}
	After the DesignBuilder model is completed, the file is then converted to \textit{.idf} file and being further processed.\\
	Firstly the standard simulation, then the dynamic simulation is performed
	\begin{itemize}
		\item Standard Schedule Simulation
		\begin{itemize}
			\item Export to \textit{.idf} files and apply SIA standard \textit{nominal schedules} in EnergyPlus
			\begin{itemize}
				\item Occupancy
				\item Appliance
				\item Electricity
				\item Lighting
				\item Ventilation
				\item Heating/Cooling
			\end{itemize}
			
			\item Undergo simulation using both \textit{2015 weather file} and \textit{standard weather file}
		\end{itemize}
		\item Dynamic Schedule Simulation
			\begin{itemize}
				\item Occupancy
				\item Appliance
				\item Electricity
				\item Lighting
				\item Ventilation
				\item Heating/Cooling
				\item Air Infiltration
			\end{itemize}
		\item Compare Standard/Dynamic Results	
	\end{itemize}
	
\section{SIA180 Building Energy Calculation}
	Recalculate the 2 buildings' energy demand using SIA180 method. 
	\begin{itemize}
		\item Obtain standard Zurich Meteostation weather data
			\begin{itemize}
				\item Average monthly temperature
				\item Heating degree days
				\item Heating days
				\item Solar Radiation onto east/west/south/north/horizontal surface
			\end{itemize}
		\item Obtain 2015 weather data by using \textit{Rhino/Grasshopper}
			\begin{itemize}
				\item Average monthly temperature
				\item Heating degree days
				\item Heating days
				\item Solar Radiation onto east/west/south/north/horizontal surface
			\end{itemize}
		\item Obtain Building Parameters
			\begin{itemize}
				\item Wall and roof properties
				\item Window properties
				\item Convection Coefficient
			\end{itemize}
		\item Calculate Energy Gains
			\begin{itemize}
				\item Solar Gain
				\item Internal Gain (from appliances)
			\end{itemize}
		\item Calculate Energy Loses
			\begin{itemize}
				\item Transmission Loss
				\item Ventilation Loss
			\end{itemize}
		\item Obtain Key Assumptions	
	\end{itemize}
	
\section{Building Envelope and Weather Condition Calibration}
	Take 10-15 days in June or September and compare the historical indoor/outdoor temperature vs. simulation indoor/weather temperatue.
	\subsection{Honggerstrasse Building Calibration}
		\begin{itemize}
			\item Date: 1 June - 10 June
			\item Compare weather file outdoor temperature with recorded site temperature
			\item Calibrate indoor temperature
				\begin{itemize}
					\item Modify appliance level and appliance schedule
					\item Modify air infiltration level
					\item Apply Shading schedule
				\end{itemize}
			\item Compare recorded annual consumption vs. calibrated annual consumption
		\end{itemize}
	
	\subsection{Sumatrastrasse Building Calibration}
		\begin{itemize}
			\item Date: 1 September - 15 September
			\item Calibrate indoor temperature
				\begin{itemize}
					\item Apply Shading schedule
					\item Modify air infiltration level
					\item Fix air ventilation setting
				\end{itemize}
			\item Compare recorded annual consumption vs. calibrated annual consumption
		\end{itemize}
		
\section{Building Parameters Variation}
	The final part of the research is to determine the key parameters that make significant influence to the simulation reslut. 

	\subsection{jE-Plus Dynamic Analysis}
		Program \textit{je-Plus} is used to assign different values to the targeted parameters. These targeted parameters include:
		\begin{itemize}
			\item Schedule and occupancy of different areas and facilities
				\begin{itemize}
					\item People activities
					\item Lighting
					\item Appliances
					\item Outdoor air supply (ventilation)
					\item Domestic hot water
				\end{itemize}
			\item Building envelope properties
				\begin{itemize}
					\item Air tightness (Infiltration)
					\item Heat convection coefficient
					\item Outside layer solar absorptance
				\end{itemize}
			\item Heating and cooling setpoint temperature
		\end{itemize}
	
	\subsection{Correlation Matrix}
		Correlation matrix is produced to compare the influence of all parameters.


\section{Urban Heat Island Effect and Global Warming Effect}
	Take local temperature (in winter) and compare it with the station temperature. Then discover a pattern and apply it onto the station weather file and generate an "urban heat island weather file"

	Draw a box plot and a histrogram to compare the effects of global warming and urban heat island effect.




\begin{comment}
	\end{multicols*}
	\end{landscape}
\end{comment}


\end{document}