\documentclass[11pt, a4paper]{article}
\usepackage{amsmath}
\usepackage{amssymb}
\usepackage{amsthm}
\usepackage{epsfig}
\usepackage[margin=0.5in]{geometry}
\usepackage{enumitem}
% \usepackage{enumerate}
\usepackage{graphicx}

%%%%%%%%%% Package for Highlight         %%%%%%%
\usepackage[usenames, dvipsnames]{xcolor}
%%%%%%%%%%%%%%%%%%%%%%%%%%%%%%%%%%%%%%%%%%%%%%%%%%


\usepackage{tikz}
\usetikzlibrary{automata, positioning}
\usepackage{caption}
\usepackage{mathtools}
\usepackage{algorithm}
\usepackage{algorithmic}
\usepackage{todonotes}
\usepackage{times}
\usepackage{pdfpages}
\usepackage{mathtools}
%\usepackage[]{algorithm2e}

\usepackage{lipsum} % paragraph spacing
\usepackage{titlesec} % paragraph spacing
\usepackage{comment}

%%%%%%%%%%% CHINESE TYPING  %%%%%%%%%%%%
\usepackage{CJK} 
% How to use: 
% \begin{CJK}{UTF8}{} 
% \end{CJK}
%%%%%%%%%%%%%%%%%%%%%%%%%%%%%%%%%%%%%%%%%%%

%%%%%%%%%%%%%%%%%%%%%%%%%%
% 3 Columns & landscape	 %
%%%%%%%%%%%%%%%%%%%%%%%%%%
%\usepackage{multicol}
%\usepackage{pdflscape}


% For writing code:
\usepackage{fancyvrb}
\DefineVerbatimEnvironment{code}{Verbatim}{fontsize=\small}
\def\showvrb#1{%
	\texttt{\detokenize{#1}}%
}

\linespread{1.0}
\setlength{\parindent}{0pt} % Sets the indent length to 0
\setlength{\parskip}{0pt plus 1pt minus 1pt} % paragraph vertical distance

\everymath{\displaystyle} % displays inline math as displaymath

\hyphenpenalty=10000 % force no hyphenation



\begin{comment}
%%%%%%%%%%%%%%%%%%%%%%%%%%%%%%%%%
% ONLY RECOMMENDED FOR 3 COLUMN %
%%%%%%%%%%%%%%%%%%%%%%%%%%%%%%%%%%%%%%%%%%%%%%%%%%%%%%%%%%%%%%%%%%%%%

\setlist[itemize]{noitemsep, topsep=0pt, leftmargin=0.2in} % compacts lists with no item separation,indent from the left side = 0.2
\titlespacing\section{1pt}{5pt plus 4pt minus 2pt}{2pt plus 2pt minus 2pt}%paragraph spacing
\titlespacing\subsection{0pt}{5pt plus 4pt minus 2pt}{2pt plus 2pt minus 2pt}%paragraph spacing
\titlespacing\subsubsection{0pt}{5pt plus 4pt minus 2pt}{2pt plus 2pt minus 2pt}%paragraph spacing
\end{comment}


%%%%%%%%%%%%%%%% Normally use this! %%%%%%%%%%%%%%%%%%%%%%%%%%%%%%%%%%%%%%%%%%%%%%%%%%%%
%
\setlist[itemize]{noitemsep, topsep=0pt} % compacts lists with no item separation    %%%
%
%
%%%%%%%%%%%%%%%%%%%%%%%%%%%%%%%%%%%%%%%%%%%%%%%%%%%%%%%%%%%%%%%%%%%%%%%%%%%%%%%%%%%%%%%%


\title{Performance Gap}
\author{Ying He}

%%%%%%%%%%%%%%%%%%%%%%%%%%%%%%%
%   Commonly used theorems    %
%%%%%%%%%%%%%%%%%%%%%%%%%%%%%%%

% Use roman for text - numbering follows onwards
\theoremstyle{definition}
\newtheorem{defn}{Definition}[section]
\newtheorem{them}{Theorem}[section]
\newtheorem{lem}{Lemma}[them]
\newtheorem{prop}[them]{Proposition}
\newtheorem{corr}[them]{Corollary}
\newtheorem{corrr}{Corollary}[them] %for consistency issues
% examples follow different numbering
\newtheorem{eg}{Example}[section]
\newtheorem{egg}[them]{Example}
\newtheorem*{themf}{Theorem 5.0}
\newtheorem*{themm}{Theorem}
\newtheorem{question}{Question}

%%%%%%%%%%%%%
% Shortcuts %
%%%%%%%%%%%%%

\newcommand{\Q}{\mathbb{Q}}
\newcommand{\C}{\mathbb{C}}
\newcommand{\F}{\mathbb{F}}
\newcommand{\R}{\mathbb{R}}
\newcommand{\Z}{\mathbb{Z}}
\newcommand{\N}{\mathbb{N}}
\newcommand{\D}{\mathbb{D}}

\newcommand{\mcC}{\mathcal{C}}
\newcommand{\mcN}{\mathcal{N}}
\newcommand{\mcE}{\mathcal{E}}
\newcommand{\mcP}{\mathcal{P}}
\newcommand{\mcF}{\mathcal{F}}
\newcommand{\mcS}{\mathcal{S}}
\newcommand{\mcA}{\mathcal{A}}
\newcommand{\mcB}{\mathcal{B}}
\newcommand{\mcX}{\mathcal{X}}
\newcommand{\mcL}{\mathcal{L}}
\newcommand{\mcH}{\mathcal{H}}
\newcommand{\mcl}{\mathcal{l}}

\newcommand{\bfe}{\mathbf{e}}
\newcommand{\mbf}{\mathbf}

\newcommand{\ran}{\mbox{Ran}}

\newcommand{\al}{\alpha}
\newcommand{\s}{\sigma}
\newcommand{\ep}{\epsilon}
\newcommand{\dl}{\delta}
\newcommand{\sg}{\sigma}

\newcommand{\bs}{\backslash}
\newcommand{\n}{\\}
\newcommand{\ol}{\overline}
%\newcommand{\bb}{\textbf} %Bold front
%\newcommand{\ul}{\underline} %underline

\newcommand{\ltlU}{\textsf{ \bfseries UNTIL }}
\newcommand{\ltlF}{\textsf{\bfseries F }}
\newcommand{\ltlG}{\textsf{\bfseries G }}
\newcommand{\ltlX}{\textsf{\bfseries X }}
\newcommand{\ltlA}{\textsf{\bfseries A }}
\newcommand{\ltlE}{\textsf{\bfseries E }}

%%%%%%%%%%%%%%%%%%%%%%%%%
% Standard big brackets %
%%%%%%%%%%%%%%%%%%%%%%%%%

\newcommand{\floor}[1]{\left\lfloor{#1}\right\rfloor}
\newcommand{\ceil}[1]{\left\lceil{#1}\right\rceil}
\newcommand{\bbA}[1]{\left({#1}\right)}
\newcommand{\bbB}[1]{\left[{#1}\right]}
\newcommand{\bbC}[1]{\left\{{#1}\right\}}



%%%%%%%%%%%%%%%%%
%  MATRIX       %
%%%%%%%%%%%%%%%%%
\newcommand{\matxx}[9]{\begin{pmatrix} #1 & #2 & #3 \\ #4 & #5 & #6 \\ #7 & #8 & #9 \end{pmatrix}}
\newcommand{\matx}[4]{\begin{pmatrix} #1 & #2 \\ #3 & #4 \end{pmatrix}}
%example
%    $\matx{1}{2}{3}{4}{5}{6}{7}{8}{9}$

%%%%%%%%%%%%%%%%

%%%%%%%%%%%%%%%%%%%%%%%%%%%%%
% Highlight color
%%%%%%%%%%%%%%%%%%%%%%%%%%%

\newcommand{\hgy}{\colorbox{GreenYellow}} % Light Green
\newcommand{\hyg}{\colorbox{YellowGreen}} % another Light Green
\newcommand{\hlb}{\colorbox{SkyBlue}} % light blue
\newcommand{\hpink}{\colorbox{pink}} % Light pink
\newcommand{\hor}{\colorbox{orange}} % Orange
\newcommand{\hlor}{\colorbox{Apricot}} % Light Orange




%%%%%%%%%%%%%%%%%%
% Start Document %
%%%%%%%%%%%%%%%%%%

\begin{document}
\begin{comment}
	\begin{landscape}
	\begin{multicols*}{3}
\end{comment}

\begin{comment}
\textbf{abcdefg}



%\tableofcontents


\hlor{abcd}

\[\matx{1}{2}{3}{4}\] al;dfjajfa

woyaozaizhedazi $\matx{1}{2}{3}{4}$



\[\alpha \cdot \beta \]



123456789/*
\end{comment}
\maketitle
\tableofcontents

\section{Abstract}

\section{Introduction}
	\subsection{Purpose of this thesis}
		The purpose of this thesis is to find out most influential factors in building simulation. By performing an indepth analysis of 2 existing buildings, the influence of all parameters can be determined.
	\subsection{Residential Building Introduction}
		\begin{itemize}
			\item Location of the building
			\item Year of construction
			\item Building plan and brief introduction (number of floors/areas/orientation/etc)
		\end{itemize}
		
	\subsection{Office Building Introduction}
		Same as Residential Building

\section{Literature Review}
	\subsection{SIA Documentations}
		Introduction of SIA 180 calculation method. Also briefly introduce a set of documentations such as 
		\begin{itemize}
			\item SIA Weather data
			\item SIA dynamic energy analysis
			\item SIA occupancy and schedule
		\end{itemize}
		

	\subsection{Previous Analyses}
		\begin{itemize}
			\item Discuss the existing result of the previous study. 
			\item A significant performance gap is observed. Possible causes of the performance gap.
			\item The data and measurement of previous studies.
		\end{itemize}
		 


\section{Methodology}
	\subsection{DesignBuilder - Building Modeling}
		A brief introduction of DesignBuilder, also describe the scope of work (Building envelope, create a formated file for EnergyPlus engine, also provide accurate geometry data for SIA calculation)
		\subsubsection{Existing Plan}
		Show the building floor plan and functions
		\subsubsection{Building Envelope Material}
		A detail description about building envelope material\\
		Show the complete building model
		\subsubsection{Assumptions}
			\begin{itemize}
				\item Wall/Roof/Ground Floor boundary conditions
				\item Adiabatic boundary conditions
			\end{itemize}
			
	\subsection{EnergyPlus on Nominal Schedule Analysis}
		How did I use EnergyPlus in this Project, and how dose it help me in presenting the data
		
		
		\subsubsection{Schedule and Occupancy Assumptions}
			\begin{itemize}
				\item Detail Schedule and Occupancy Assumptions
				\item Documentation base (SIA xxx)
			\end{itemize}
			
		\subsubsection{Weather Data Selection}
			\begin{itemize}
				\item Data source: IDAweb
				\item Weather data selection base: location,date
				\item Weather file modification (replaced parameters)
					\begin{itemize}
						\item Temperature (dry/wet)
						\item Relative Humidity
						\item Wind speed
						\item Wind direction
					\end{itemize}
			\end{itemize}
			

	\subsection{SIA Calculation}
	Briefly describe the reason why another SIA calculation is necessary\\
	(Huge gap in building area assumption, huge gap in weather condition)


		\subsubsection{Basic description of SIA Calculation Method}
			\begin{itemize}
				\item Gain
				\begin{itemize}
					\item Solar Gain
					\item Internal Gain (appliance/activities)
				\end{itemize}
				\item Loss
				\begin{itemize}
					\item Transmission Loss (Conduction and Convection)
					\item Ventilation Loss (Air circulation and infiltration)
				\end{itemize}
			\end{itemize}

		\subsubsection{Calculation Assumptions}	
			Two sets of calculation are performed. One uses SIA standard weather condition, the other one uses 2015 weather condition. The 2015 weather data is extracted from the 2015 weather file using Rhino/Grasshopper add-on.
		
		\subsubsection{Measurements expected to close the performance gap}
			Mention a number of suspected parameters which help to help the calculate result reach the historical measurement value.
			\begin{itemize}
				\item Modify the infiltration value
				\item Use correct weather file
				\item Use more accurate floor area
			\end{itemize}

	\subsection{jEPlus Dynamic Simulation}
		Why using je-Plus and how to use
		\subsubsection{Basic description of je-Plus}
		\begin{itemize}
			\item Main functions
			\item Advantages
			\item How to use
		\end{itemize}
	
		\subsubsection{Dynamic Parameter Assumptions}
		List of parameters, range of parameters and distribution of parameters


	\subsection{Calibration}
		How to calibrate the building envelope by choosing summer period (early autumn) and compare the building indoor environment with the historical data.

		\subsubsection{Outdoor Environment Calibration}
		\subsubsection{Building Envelope Calibration}
		\begin{itemize}
			\item EnergyPlus Hourly Analysis
		\end{itemize}


	
	\subsection{Data Processing}
		Briefly describe the resulting data structure from je-Plus and introduce 2 different graphs.
		\subsubsection{Dynamic Analysis Range}

		\subsubsection{Correlation Matrix}
		An introduction about Correlation matrix

\section{Results}

	\subsection{Initial Annual Energy Analysis}
		
		\subsubsection{EnergyPlus Simulation Result}		
		\begin{itemize}
			\item Floor Area (be used in SIA calculation)
			\item Heating Demand
			\item Air Ventilation
		\end{itemize}
		The first set of dynamic analysis (4 sets of heating demand distribution from infiltration 0.1 to 0.4)

		\subsubsection{SIA Calculation}
			Take
		\subsubsection{Calibration}
			\begin{itemize}
				\item Steps of calibration (hourly - annually)
				\item Results of each variation (new lighting schedule/shading schedule etc)
				\item Recommended base values
			\end{itemize}
			
		\subsubsection{jE-Plus Simulation Results}
			\begin{itemize}
				\item Dynamic Heating Demand Variation
				\item Dynamic DHW Demand Variation
				\item Results of all parameters (range and distribution of heating demand and DHW demand)
				\item Correlation of parameters
			\end{itemize}



\section{Discussion}

	\begin{itemize}
		\item Key parameters (Which parameters are the most important and which are not as important)
		\item Key assumptions (Are these assumption still applicable)
		\item Recommendation (building envelope much be accurate, a weather data update is critical etc)
	\end{itemize}
			
\section{Conclusion}
	\begin{itemize}
		\item Key parameters
		\item Recommended set of parameters
		\item General Recommendations
	\end{itemize}
	












\begin{comment}
	\end{multicols*}
	\end{landscape}
\end{comment}


\end{document}